\chapter{Trabalhos Relacionados}
%%
%%\textcolor{red}{Linha escalonadores distribuidos: revisao do Wilton}
%%
%%MapReduce, Spark, XYZ
%%Conclusão: a minha ideia é que pode ser usado qualquer escalonador, desde que seja possível representar como microsserviço (stateless).
%%
%%\textcolor{red}{Tecer comentários sobre TF: replicas ja foram usadas para contornar falhas. %%\url{https://gkoslovski.github.io/files/ijguc-2019.pdf}}
%%
%%Tabela de conclusao. Colunas: 1 ref, 2 foco, 3 container/VM/tarefa. Mostrar que microsservico é uma opção viavel.
%%
%%A área de escalonamento possui diversas referencias fortes em diversas áreas. Entretanto, a conteinerização é um termo recente comparado a toda história 

A área de escalonamento de tarefas é estudada a décadas, considera-se um ramo com intersecção em diferentes campos da ciência da computação. Um dos principais fatores que impulsionam a intersecção entre as áreas está na natureza do problema de escalonamento, pois na maioria dos casos resolver um problema de escalonamento reflete em otimizar um problema \textit{NP-difícil}. Dessa forma, na literatura, encontra-se diferentes pesquisas que refletem em soluções heurísticas e refinadas, muitas vezes relacionadas com a área de inteligência artificial. Por se tratar da solução de um problema \textit{np-difícil}, a solução computada basta ser boa o suficiente para um cenário específico, sendo assim, a solução encontrada nem sempre é refletida na ótima global.

Esta seção analisará trabalhos recentes da literatura que possuem relação com escalonamento de contêineres. Por ainda se tratar de um recorte amplo na área de escalonamento, aqui analisou-se diferentes trabalhos que atacam diferentes características, seja relacionado com otimização energética em \textit{data centers} como também trabalhos que buscaram otimizar o desempenho de escalonamento.

\section{Redução do custo energético}
No trabalho de \citeonline{sureshkumar2017optimizing} os autores propuseram um algoritmo de escalonamento de contêineres, para a tecnologia \textit{docker}, baseado em balanceamento de carga. O algoritmo consiste em um limiar calculado a partir da sobrecarga do \textit{cluster}, o objetivo do limiar é que as cargas dos contêineres não sejam muito altas e baixas. Quando a carga ultrapassa o limiar, um novo contêiner é criado no balanceamento de carga. Por outro lado, quando a carga é muito baixa, o contêiner é destruído com o objetivo de economizar custo energético.

\section{Otimização multi objetivo}
Em \citeonline{liu2018new} os autores desenvolveram um novo algoritmo de escalonamento de contêineres denominado \textit{multipot}. Neste projeto foi estudado múltiplos critérios para a seleção de um \textit{node} para provisionar o contêiner. O algoritmo considera cinco métricas chaves:
\begin{enumerate}
	\item Uso de CPU de cada \textit{node};
	\item Uso de memória de cada \textit{node};
	\item Tempo da transmissão da imagem do contêiner na internet;
	\item Associação entre os contêineres e os \textit{nodes};
	\item Agrupamento de contêineres.
\end{enumerate}
Todas essas métricas foram consideradas, pois, de acordo com os autores, afetam no desempenho das aplicações que estão sendo executadas pelos contêineres. A função objetivo de escalonamento é a otimização da composição de todas as métricas chaves, ou seja, para cada métrica chave é relacionado um escore, esses escores são agrupados em uma função de composição que representa a função objetivo.

Em \citeonline{menouer2019new} foi desenvolvido uma nova estratégia de escalonamento baseado em um algoritmo de decisão multi-critério. A abordagem consiste em escalonar contêineres baseado em três critérios que estão relacionados a todos os \textit{nodes}  que constitui a infraestrutura de nuvem: (1) O número de contêineres em execução; (2) A disponibilidade de CPU; (3) A disponibilidade do espaço de memória.

Em \citeonline{menouer2019power} os autores utilizaram técnicas refinadas de \textit{machine learning} em um ambiente de nuvem para construção de um escalonador de contêineres. O objetivo dessa abordagem é reduzir o consumo energético de infraestruturas de nuvens heterogêneas. O principio corresponde a dois passos denominados (1) aprender e (2) escalonar, que são aplicados a cada novo contêiner que é submetido à plataforma. No passo (1) é estimado o consumo energético de cada \textit{node}, logo, é elencado grupos de \textit{nodes} que formam uma estrutura de nuvem heterogênea em um \textit{cluster} de acordo com o seu consumo energético.
Já no passo (2), é selecionado o \textit{node} que corresponde  ao menor consumo energético. O algoritmo foi implementado para a plataforma \textit{docker swarm}.

\section{Considerações parciais}
Está seção elencou alguns do trabalhos encontrados na literatura acerca do tema escalonamento de contêineres. Nota-se que os trabalhos envolvidos possuem definição de parâmetros e métricas de otimização, como por exemplo: consumo energético, desempenho. Por se tratar de um recorte grande na literatura, o tema escalonamento abre espaço para implementação de diferentes soluções para o problema. Todavia, os trabalhos relacionados aqui elencados não consideram o fator escalabilidade, não há nenhuma solução distribuída. A escassez de projetos de escalonamento de contêineres que envolvam o desenvolvimento de uma arquitetura distribuída, é uma das principais motivações para o presente trabalho.
 

