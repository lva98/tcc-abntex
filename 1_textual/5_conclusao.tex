\chapter{Conclusão}

Escalonamento é considerado um processo de tomada de decisão, em resumo, no contexto do presente trabalho, o objetivo é alocar contêineres em nós computacionais denominados \textit{nodes} com recursos finito. Há diversos parâmetros a ser considerados, por consequência, é possível alcançar as mais divergentes funções objetivos -- reduzir consumo energético, otimizar utilização de recursos do \textit{data center}. Logo, de maneira geral, os algoritmos de otimização de escalonamento se enquadram na classe \textit{NP-difícil} \cite{ullman1975np}, visto que, a natureza do problema de escalonamento é combinatória.

A explosão da demanda de poder computacional dos \textit{data centers} uniram, ainda mais, as áreas entre escalonamento e sistemas distribuídos. O objetivo desse trabalho, é, a partir da intersecção entre as duas áreas, resolver de forma elegante o problema de escalonamento para \textit{data centers} de larga escala. De acordo com a literatura, \textit{data centers} de larga escala necessitam de componentes internos sofisticados e adaptáveis que lidam com o gerenciamento de centenas de milhares de cargas de trabalho. Assim, a construção de uma arquitetura de escalonamento distribuída, até o momento, é a solução para que as empresas, como \textit{Google} e \textit{Microsoft}, tratem um grande volume de requisições de escalonamento \cite{Wang2019Pigeon, Google2015Borg}.

Ao unir as áreas entre sistemas distribuídos e escalonamento encontra-se quantidade significativa de estudos explorados, entretanto, a intersecção entre essas áreas junto com orquestradores de contêineres é escasso na literatura como visto no Capítulo 3. Este trabalho possui como objetivo suprir essa carência, uma vez que a principal arquitetura de escalonamento encontrada em \textit{data centers} de larga escala, atualmente, é distribuída. Para validação da proposta, os resultados da arquitetura aqui desenvolvida serão comparados com o escalonador padrão do \textit{Kubernetes}. Com isso, objetiva-se chegar no resultado positivo, cujo as métricas de tempo de espera de escalonamento e \textit{makespan} sejam reduzidas. Além disso, haverá um estudo relacionado ao comportamento da métrica de \textit{slowdown} nos dois cenários propostos: escalonamento padrão e distribuído.


\section{Trabalhos futuros}
\noindent
A seguir é exposto as etapas de desenvolvimento do projeto, considerando em \textbf{\textcolor{green}{verde}} as etapas que já foram concluída no período esperado e em \textbf{preto} as etapas a ser desenvolvidas. 

\begin{enumerate}
	\item Revisão bibliográfica sobre contêineres e escalonamento;
	\item Estudo sobre orquestração de contêineres;
	\item Elaboração teórica do algoritmo de escalonamento proposto;
	\item Implementação do algoritmo desenvolvido no gerenciador de contêiner;
	
	\item Redação TCC-I;
	\item Ajustes de implementação;
	\item Desenvolver definições de análise;
	\item Análise e comparação dos resultados; e
	\item Redação TCC-II.
\end{enumerate}
% Please add the following required packages to your document preamble:
% \usepackage{multirow}
% \usepackage[table,xcdraw]{xcolor}
% If you use beamer only pass "xcolor=table" option, i.e. \documentclass[xcolor=table]{beamer}
% Please add the following required packages to your document preamble:
% \usepackage{multirow}
% \usepackage[table,xcdraw]{xcolor}
% If you use beamer only pass "xcolor=table" option, i.e. \documentclass[xcolor=table]{beamer}
\begin{table}[h]
\centering
	\caption{\label{tab:calendario_cronograma}Calendário cronograma}
	\begin{tabular}{|c|cccccccccccc|cccccccccccc|}
		\hline
		& \multicolumn{12}{c|}{2021/2}                                                                                                                                                                                                                                                                                                                                                                                                                                                                                                                     & \multicolumn{12}{c|}{2022/1}                                                                                                                                                                                                                                                                                                                                                                                                                                                                                                                     \\ \cline{2-25} 
		\multirow{-2}{*}{Etapas} & \multicolumn{2}{c|}{O}                                                & \multicolumn{2}{c|}{N}                                                                        & \multicolumn{2}{c|}{D}                                                                        & \multicolumn{2}{c|}{J}                                                                        & \multicolumn{2}{c|}{F}                                                                        & \multicolumn{2}{c|}{M}                                                   & \multicolumn{2}{c|}{A}                                                                        & \multicolumn{2}{c|}{M}                                                                        & \multicolumn{2}{c|}{J}                                                                        & \multicolumn{2}{c|}{J}                                                                        & \multicolumn{2}{c|}{A}                                                                        & \multicolumn{2}{c|}{S}                           \\ \hline
		1                        & \multicolumn{1}{c|}{} & \multicolumn{1}{c|}{\cellcolor[HTML]{009901}} & \multicolumn{1}{c|}{\cellcolor[HTML]{009901}} & \multicolumn{1}{c|}{\cellcolor[HTML]{009901}} & \multicolumn{1}{c|}{\cellcolor[HTML]{009901}} & \multicolumn{1}{c|}{}                         & \multicolumn{1}{c|}{}                         & \multicolumn{1}{c|}{}                         & \multicolumn{1}{c|}{}                         & \multicolumn{1}{c|}{}                         & \multicolumn{1}{c|}{}                         &                          & \multicolumn{1}{c|}{}                         & \multicolumn{1}{c|}{}                         & \multicolumn{1}{c|}{}                         & \multicolumn{1}{c|}{}                         & \multicolumn{1}{c|}{}                         & \multicolumn{1}{c|}{}                         & \multicolumn{1}{c|}{}                         & \multicolumn{1}{c|}{}                         & \multicolumn{1}{c|}{}                         & \multicolumn{1}{c|}{}                         & \multicolumn{1}{c|}{}                         &  \\ \hline
		2                        & \multicolumn{1}{c|}{} & \multicolumn{1}{c|}{}                         & \multicolumn{1}{c|}{\cellcolor[HTML]{009901}} & \multicolumn{1}{c|}{\cellcolor[HTML]{009901}} & \multicolumn{1}{c|}{\cellcolor[HTML]{009901}} & \multicolumn{1}{c|}{\cellcolor[HTML]{009901}} & \multicolumn{1}{c|}{}                         & \multicolumn{1}{c|}{}                         & \multicolumn{1}{c|}{}                         & \multicolumn{1}{c|}{}                         & \multicolumn{1}{c|}{}                         &                          & \multicolumn{1}{c|}{}                         & \multicolumn{1}{c|}{}                         & \multicolumn{1}{c|}{}                         & \multicolumn{1}{c|}{}                         & \multicolumn{1}{c|}{}                         & \multicolumn{1}{c|}{}                         & \multicolumn{1}{c|}{}                         & \multicolumn{1}{c|}{}                         & \multicolumn{1}{c|}{}                         & \multicolumn{1}{c|}{}                         & \multicolumn{1}{c|}{}                         &  \\ \hline
		3                        & \multicolumn{1}{c|}{} & \multicolumn{1}{c|}{}                         & \multicolumn{1}{c|}{}                         & \multicolumn{1}{c|}{\cellcolor[HTML]{009901}} & \multicolumn{1}{c|}{\cellcolor[HTML]{009901}} & \multicolumn{1}{c|}{\cellcolor[HTML]{009901}} & \multicolumn{1}{c|}{\cellcolor[HTML]{009901}} & \multicolumn{1}{c|}{\cellcolor[HTML]{009901}} & \multicolumn{1}{c|}{\cellcolor[HTML]{343434}} & \multicolumn{1}{c|}{}                         & \multicolumn{1}{c|}{}                         &                          & \multicolumn{1}{c|}{}                         & \multicolumn{1}{c|}{}                         & \multicolumn{1}{c|}{}                         & \multicolumn{1}{c|}{}                         & \multicolumn{1}{c|}{}                         & \multicolumn{1}{c|}{}                         & \multicolumn{1}{c|}{}                         & \multicolumn{1}{c|}{}                         & \multicolumn{1}{c|}{}                         & \multicolumn{1}{c|}{}                         & \multicolumn{1}{c|}{}                         &  \\ \hline
		4                        & \multicolumn{1}{c|}{} & \multicolumn{1}{c|}{}                         & \multicolumn{1}{c|}{}                         & \multicolumn{1}{c|}{}                         & \multicolumn{1}{c|}{}                         & \multicolumn{1}{c|}{}                         & \multicolumn{1}{c|}{}                         & \multicolumn{1}{c|}{\cellcolor[HTML]{FFFFFF}} & \multicolumn{1}{c|}{\cellcolor[HTML]{343434}} & \multicolumn{1}{c|}{\cellcolor[HTML]{343434}} & \multicolumn{1}{c|}{\cellcolor[HTML]{343434}} & \cellcolor[HTML]{343434} & \multicolumn{1}{c|}{\cellcolor[HTML]{343434}} & \multicolumn{1}{c|}{\cellcolor[HTML]{343434}} & \multicolumn{1}{c|}{\cellcolor[HTML]{343434}} & \multicolumn{1}{c|}{\cellcolor[HTML]{343434}} & \multicolumn{1}{c|}{}                         & \multicolumn{1}{c|}{}                         & \multicolumn{1}{c|}{}                         & \multicolumn{1}{c|}{}                         & \multicolumn{1}{c|}{}                         & \multicolumn{1}{c|}{}                         & \multicolumn{1}{c|}{}                         &  \\ \hline
		5                        & \multicolumn{1}{c|}{} & \multicolumn{1}{c|}{\cellcolor[HTML]{009901}} & \multicolumn{1}{c|}{\cellcolor[HTML]{009901}} & \multicolumn{1}{c|}{\cellcolor[HTML]{009901}} & \multicolumn{1}{c|}{\cellcolor[HTML]{009901}} & \multicolumn{1}{c|}{\cellcolor[HTML]{009901}} & \multicolumn{1}{c|}{\cellcolor[HTML]{009901}} & \multicolumn{1}{c|}{\cellcolor[HTML]{009901}} & \multicolumn{1}{c|}{\cellcolor[HTML]{FFFFFF}} & \multicolumn{1}{c|}{\cellcolor[HTML]{FFFFFF}} & \multicolumn{1}{c|}{\cellcolor[HTML]{FFFFFF}} & \cellcolor[HTML]{FFFFFF} & \multicolumn{1}{c|}{}                         & \multicolumn{1}{c|}{}                         & \multicolumn{1}{c|}{}                         & \multicolumn{1}{c|}{}                         & \multicolumn{1}{c|}{}                         & \multicolumn{1}{c|}{}                         & \multicolumn{1}{c|}{}                         & \multicolumn{1}{c|}{}                         & \multicolumn{1}{c|}{}                         & \multicolumn{1}{c|}{}                         & \multicolumn{1}{c|}{}                         &  \\ \hline
		6                        & \multicolumn{1}{c|}{} & \multicolumn{1}{c|}{}                         & \multicolumn{1}{c|}{}                         & \multicolumn{1}{c|}{}                         & \multicolumn{1}{c|}{}                         & \multicolumn{1}{c|}{}                         & \multicolumn{1}{c|}{}                         & \multicolumn{1}{c|}{}                         & \multicolumn{1}{c|}{}                         & \multicolumn{1}{c|}{}                         & \multicolumn{1}{c|}{}                         &                          & \multicolumn{1}{c|}{}                         & \multicolumn{1}{c|}{}                         & \multicolumn{1}{c|}{}                         & \multicolumn{1}{c|}{}                         & \multicolumn{1}{c|}{\cellcolor[HTML]{343434}} & \multicolumn{1}{c|}{\cellcolor[HTML]{343434}} & \multicolumn{1}{c|}{\cellcolor[HTML]{343434}} & \multicolumn{1}{c|}{}                         & \multicolumn{1}{c|}{}                         & \multicolumn{1}{c|}{}                         & \multicolumn{1}{c|}{}                         &  \\ \hline
		7                        & \multicolumn{1}{c|}{} & \multicolumn{1}{c|}{}                         & \multicolumn{1}{c|}{}                         & \multicolumn{1}{c|}{}                         & \multicolumn{1}{c|}{}                         & \multicolumn{1}{c|}{}                         & \multicolumn{1}{c|}{}                         & \multicolumn{1}{c|}{}                         & \multicolumn{1}{c|}{}                         & \multicolumn{1}{c|}{}                         & \multicolumn{1}{c|}{}                         &                          & \multicolumn{1}{c|}{}                         & \multicolumn{1}{c|}{}                         & \multicolumn{1}{c|}{}                         & \multicolumn{1}{c|}{}                         & \multicolumn{1}{c|}{}                         & \multicolumn{1}{c|}{}                         & \multicolumn{1}{c|}{\cellcolor[HTML]{343434}} & \multicolumn{1}{c|}{\cellcolor[HTML]{343434}} & \multicolumn{1}{c|}{\cellcolor[HTML]{343434}} & \multicolumn{1}{c|}{}                         & \multicolumn{1}{c|}{}                         &  \\ \hline
		8                        & \multicolumn{1}{c|}{} & \multicolumn{1}{c|}{}                         & \multicolumn{1}{c|}{}                         & \multicolumn{1}{c|}{}                         & \multicolumn{1}{c|}{}                         & \multicolumn{1}{c|}{}                         & \multicolumn{1}{c|}{}                         & \multicolumn{1}{c|}{}                         & \multicolumn{1}{c|}{}                         & \multicolumn{1}{c|}{}                         & \multicolumn{1}{c|}{}                         &                          & \multicolumn{1}{c|}{}                         & \multicolumn{1}{c|}{}                         & \multicolumn{1}{c|}{}                         & \multicolumn{1}{c|}{}                         & \multicolumn{1}{c|}{}                         & \multicolumn{1}{c|}{}                         & \multicolumn{1}{c|}{}                         & \multicolumn{1}{c|}{\cellcolor[HTML]{343434}} & \multicolumn{1}{c|}{\cellcolor[HTML]{343434}} & \multicolumn{1}{c|}{\cellcolor[HTML]{343434}} & \multicolumn{1}{c|}{}                         &  \\ \hline
		9                        & \multicolumn{1}{c|}{} & \multicolumn{1}{c|}{}                         & \multicolumn{1}{c|}{}                         & \multicolumn{1}{c|}{}                         & \multicolumn{1}{c|}{}                         & \multicolumn{1}{c|}{}                         & \multicolumn{1}{c|}{}                         & \multicolumn{1}{c|}{}                         & \multicolumn{1}{c|}{}                         & \multicolumn{1}{c|}{}                         & \multicolumn{1}{c|}{}                         &                          & \multicolumn{1}{c|}{\cellcolor[HTML]{343434}} & \multicolumn{1}{c|}{\cellcolor[HTML]{343434}} & \multicolumn{1}{c|}{\cellcolor[HTML]{343434}} & \multicolumn{1}{c|}{\cellcolor[HTML]{343434}} & \multicolumn{1}{c|}{\cellcolor[HTML]{343434}} & \multicolumn{1}{c|}{\cellcolor[HTML]{343434}} & \multicolumn{1}{c|}{\cellcolor[HTML]{343434}} & \multicolumn{1}{c|}{\cellcolor[HTML]{343434}} & \multicolumn{1}{c|}{\cellcolor[HTML]{343434}} & \multicolumn{1}{c|}{\cellcolor[HTML]{343434}} & \multicolumn{1}{c|}{\cellcolor[HTML]{343434}} &  \\ \hline
	\end{tabular}
\end{table}
