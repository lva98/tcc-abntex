\chapter{Análise Experimental}
Nesta seção objetiva-se definir os detalhes de infraestrutura do ambiente de testes, isso inclui as versões dos sistemas que serão utilizados, tanto o \textit{Kubernetes} quanto sistema operacional das unidades computacionais. Outro ponto a salientar é a definição dos parâmetros e as métricas de interesse, como também a carga de trabalho que irá estressar a arquitetura no ambiente de testes.

\section{Métricas e Parâmetros}
Um dos principais desafios do presente trabalho é a investigação e definição dos parâmetros da arquitetura proposta. A princípio, foram analisados dois parâmetros: (1) variação no número de réplicas do componente \textit{Worker} e (2) quantidade de \textit{nodes} que cada \textit{Worker} gerenciará o escalonamento. Os parâmetros são diretamente proporcionais, pois ao variar a quantidade de réplicas será também variado a número de \textit{nodes} vinculado a cada \textit{Worker}. Por exemplo, considere um \textit{cluster} com 8 \textit{nodes} e 2 réplicas \textit{workers}, cada réplica será responsável por executar o escalonamento em uma partição do \textit{cluster} com 4 \textit{nodes}. O cálculo feito para o tamanho de cada partição é a razão entre a quantidade de \textit{nodes} e réplicas do \textit{Worker}.


As métricas de interesse do presente trabalho estão relacionadas com o desempenho de escalonamento em cenários de falhas. Quanto menos tempo uma carga de trabalho permanecer na fila, maior é o desempenho da proposta de escalonamento. Portanto, há duas métricas que serão avaliadas de acordos com os parâmetros propostos: tempo de espera de escalonamento e \textit{makespan}. As métricas de sobrecarga também serão importantes, principalmente relacionadas a sobrecarga do \textit{node}, seja de espaço ou processamento.

%Um dos principais desafios do presente trabalho é investigar e definir os parâmetros da arquitetura proposta. A princípio, há dois parâmetros inicias os quais serão desenvolvidos: (1) variação no número de réplicas do escalonador e (2) quantidade de \textit{nodes} do \textit{cluster} que cada \textit{worker} gerenciará. No contexto do presente trabalho, (1) será variado em diversas rodadas de testes, será considerado de início apenas 2 réplicas e subir a quantidade gradativamente até atingir o limite físico da arquitetura. Já (2) possui um nível de complexidade maior que (1), a forma natural é dividir de forma proporcional, por exemplo, em um \textit{cluster} com 10 \textit{nodes} e 5 \textit{workers} então cada \textit{worker} gerenciará 2 \textit{nodes}. Contudo, de acordo com a literatura, esse é um parâmetro que há margem de refinamento. Considera-se o trabalho de \citeonline{vaucher2018sgxaware}, o qual consiste em escalonamento em \textit{cluster} heterogêneo -- máquinas com diferentes arquiteturas e capacidades de processamento. Neste cenário, não seria interessante utilizar o método proporcional de divisão entre \textit{workers} e \textit{nodes}. Outro exemplo é o trabalho de \citeonline{Wang2019Pigeon}, o qual consiste em particionar o \textit{cluster} de acordo com a complexidade das cargas de trabalho. Este trabalho sugere que cargas de trabalho menores devem ser executadas em um segmento do \textit{cluster} diferente que cargas de trabalhos consideradas complexas. Neste cenário houve acréscimo do desempenho na métrica de tempo de espera.

%As métricas de interesse do presente trabalho estão relacionadas com o desempenho de escalonamento em cenários de falhas. Quanto menos tempo uma carga de trabalho permanecer na fila, maior é o desempenho da proposta de escalonamento. Portanto, há duas métricas que serão avaliadas de acordos com os parâmetros propostos: tempo de espera de escalonamento e \textit{makespan}. As métricas de sobrecarga também serão importantes, principalmente relacionadas a sobrecarga do \textit{node}, seja de espaço ou processamento.

\section{Cenário experimental}
Para o ambiente de testes será utilizada a Nuvem da UDESC, com isso, os \textit{nodes} refletirão em máquinas virtuais gerenciadas pelo \textit{Openstack}. Quanto ao sistema operacional não há muita influência no decorrer do presente trabalho, pois o único requisito do \textit{Kubernetes} na versão \textit{v1.23} é que o \textit{node} seja executado em ambiente \textit{linux}, portanto, o \ac{SO} escolhido será \textit{ubuntu cloud} que é projetado para nuvem. A princípio, os \textit{nodes} são máquinas virtuais com recurso limitado em 4 núcleos e 4 gigas de memória \textit{RAM}. No total, o ambiente experimental possui 16 núcleos e 16 gigas de \textit{RAM}.

A configuração padrão do \ac{KMS} no ambiente de teste irá contar com 2 réplicas do componente \textit{Master} e também 2 réplicas \textit{Worker}. O objetivo do trabalho é analisar o desempenho em cenário de falhas de escalonamento, portanto, o baixo número de réplicas não influenciou nos resultados. Pois os testes consistiram em derrubar de forma parcial algum componente principal do \ac{KMS}. Além das réplicas dos componentes internos, o banco de dados em memória \textit{Redis} também será distribuído e contará com alta disponibilidade. Portanto, todos os módulos do sistema estão protegidos por alguma técnica de controle de falhas.

\section{Protocolo Experimental}

O protocolo experimental é considerado um simulador de eventos discretos, possui como parâmetro o total de cargas de trabalhos que serão submetidas à plataforma e o tempo total de execução. As cargas serão submetidas de forma gradual respeitando a curva da distribuição normal, ou seja, o pico de submissão será exatamente na metade do tempo total de execução. Além disso, haverá também um simulador de eventos de erro, em síntese, consiste em simular erros de \textit{crash} nos sistemas de escalonamentos derrubando os processos. Os eventos de erro também serão disparados respeitando a curva da distribuição normal, isto é, o pico de disparo de erros será na metade do tempo total de execução da bateria de testes.

%TODO explicar o algoritmo de simulador de eventos
%TODO pseudocódigo
%TODO definir um profile fixo de disparo de erros

\section{Resultados}





















