% resumo em português
\setlength{\absparsep}{18pt} % ajusta o espaçamento dos parágrafos do resumo
\begin{resumo}
    Na contemporaneidade, nota-se uma tendência no desenvolvimento de sistemas distribuídos, a qual é impulsionada pela conteinerização de microsserviços provisionados em \textit{data centers}. Especificamente, contêiner é uma tecnologia responsável por encapsular o ambiente da aplicação abstraindo particularidades de sistema operacional, bibliotecas e configurações. Em plataformas de computação atuais (\textit{e.g.}, \textit{data centers}, nuvem) essa tecnologia é implantada em larga escala, pois o contêiner empacota aplicações com o objetivo de otimizar o uso de recurso computacional da máquina hospedeira. O \textit{Kubernetes} é um orquestrador de contêineres responsável por criar, gerenciar e escalar microsserviços na forma de contêineres. O escalonador padrão do \textit{Kubernetes} atua apenas nas estratégias de espalhamento ou agrupamento de contêineres. Para reparar essa deficiência, na literatura encontram-se técnicas de escalonamento que refletem em redução de consumo energético, uso total de recursos ou equidade entre usuários. Entretanto, grande parte desses estudos não consideram o desenvolvimento de um algoritmo escalável. O escalonador baseado em arquitetura distribuída fornece otimização do uso dos recursos do \textit{data center}, como também, melhor tolerância a falhas, por consequência, diminuição considerável do tempo de espera de escalonamento em cenário de falhas. Dessa forma, o presente trabalho visou o desenvolvimento de um escalonador distribuído baseado em microsserviços para \textit{Kubernetes}, a partir do resultados observou-se um bom desempenho da proposta quando comparado com a abordagem padrão de escalonamento do \textit{Kubernetes}.
    
    \textbf{Palavras-chave}: Sistemas distribuídos, microsserviços, contêiner, \textit{Kubernetes}, escalonamento.
\end{resumo}

% resumo em inglês
\begin{resumo}[Abstract]
 \begin{otherlanguage*}{english}
   Nowdays there is a tendency in the development of distributed systems, which is driven by the containerization of microservices provisioned in data centers. Specifically, a container is a technology that encapsulate the application environment, abstract particularities of the operating system, libraries and configurations. In actual computing platforms (\textit{e.g.}, data centers, cloud) this technology is deployed on a large scale, because the container packages applications with the aim of optimizing the use of the host machine's computational resources. The \textit{Kubernetes} is a container orchestrator responsible for creating, managing and scaling microservices represented by containers. The default scheduler of \textit{Kubernetes} only works on spreading or container binpacking strategies. In the literature there are scheduling techniques that reflect in energy consumption reduction, total use of resources or equity between users. But, most of these studies do not consider the development of a scalable algorithm. The scheduler based on distributed architecture provides optimization at the use of \textit{data center} resources, better fault tolerance, consequently, a considerable reduction in the scheduling waiting time in a failure scenario. Therefore, this study aims to develop a distributed scheduler based on microservices for \textit{Kubernetes}, thus increasing scalability and reducing scheduling waiting time and \textit{makespan} metrics.
   \vspace{\onelineskip}
 
   \noindent 
   \textbf{Keywords}: Distributed Systems, microservices, container, \textit{Kubernetes}, scheduling.
 \end{otherlanguage*}
\end{resumo}
% ---